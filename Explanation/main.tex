\documentclass[10pt, a4paper]{article}
\usepackage[margin = 1.5in]{geometry}
\usepackage[utf8]{inputenc}
\usepackage[fleqn]{amsmath} % fleqn: align multiline equations left
\usepackage{amsfonts}
\usepackage{enumerate}

% code block
\usepackage{listings, lstautogobble}
\usepackage{xcolor}
\usepackage{mdframed} 
\lstset{basicstyle=\small\ttfamily, columns=flexible, breaklines=true, autogobble=true}
\definecolor{light-gray}{gray}{0.95}

\newcommand{\opencodeblock}{\begin{mdframed}[backgroundcolor=light-gray, roundcorner=10pt,leftmargin=1, rightmargin=1, innerleftmargin=15, innertopmargin=15,innerbottommargin=15, outerlinewidth=1, linecolor=light-gray] \begin{lstlisting}}
\newcommand{\closecodeblock}{\end{lstlisting} \end{mdframed}}

\newcommand{\beginproof}{\vspace{2mm} \textbf{Proof:} \vspace{2mm}}
\newcommand{\qed}{\vspace{2mm} \begin{flushright} \textbf{QED} \end{flushright} \vspace{2mm}}
\newcommand{\awnswer}[1]{\underline{\underline{\textbf{#1}}}}

\title{Algorithm explanation}

\begin{document}

\maketitle

\begin{enumerate}
	\item Algorithm X

	      Algorithm X finds rows from a matrix such that those rows contain a 1 in each column

	      For example, if algorithm X runs on the following matrix:

	      \begin{lstlisting}
			rows  matrix
			A    | 1 0 0 1 0 0 1
			B    | 1 0 0 1 0 0 0
			C    | 0 0 0 1 1 0 1
			D    | 0 0 1 0 1 1 0
			E    | 0 1 1 0 0 1 1
			F    | 0 1 0 0 0 0 1
		  \end{lstlisting}

	      Algorithm X should return rows B, D, and F, because:

	      \begin{lstlisting}
			rows  matrix
			B    | 1 0 0 1 0 0 0
			D    | 0 0 1 0 1 1 0
			F    | 0 1 0 0 0 0 1
			sum  | 1 1 1 1 1 1 1
		  \end{lstlisting}

	      This is the pseudocode for algorithm X:

	      \begin{mdframed}[backgroundcolor=light-gray, roundcorner=10pt,leftmargin=1, rightmargin=1, innerleftmargin=15, innertopmargin=15,innerbottommargin=15, outerlinewidth=1, linecolor=light-gray]
		      \begin{lstlisting}
				If matrix A only contains 1s, the current solution is a valid solution
				Thus, return the list solutions
				
				If matrix A is empty this is not a valid solution, so terminate the branch

				Choose column c such that it contains the least number of 1s

				For each row r such that A[r][c] = 1 (row r contains a 1 in column r)
					Include row r in the solution.

					For each column j such that A[r][j] = 1
						Add column j to the list of columns that have to be removed
						For each row i such that A[i][j] = 1
							Add row i to the list of rows that have to be removed
					
					Remove the rows and columns that have to be removed 
					
					Repeat the algorithm recursively on matrix A

					If the branch finds an valid solution add the solution to the list of solutions
					Else, remove row r from the solution because it is not a valid solution
				
				This is not a valid solution, so terminate the branch

			\end{lstlisting}
	      \end{mdframed}

	\item Matrix Creator

	      In order to solve the pentomino puzzle using algorithm x, we have to create a matrix that contains all the possible placements of the pentominos.

	      The matrix has to look look this:

	      \begin{tabular}{ c | c  c  c  c   c | c  c  c c c c c c}
		      column & 0 & 1 & ... & 9 & 11 & 12 & 13 & 14 & ... \\
		      \hline
		      data   & 1 & 0 &     & 0 & 0  & 1  & 1  & 0        \\
		             & 1 & 0 &     & 0 & 0  & 0  & 1  & 1        \\
		             & 0 & 1 &     & 0 & 0  & 1  & 1  & 0        \\
		             & 0 & 1 &     & 0 & 0  & 0  & 0  & 0        \\
		             & 0 & 0 &     & 1 & 0  & 1  & 0  & 0        \\
		             & 0 & 0 &     & 1 & 0  & 0  & 0  & 1        \\
		             & 0 & 0 &     & 0 & 1  & 1  & 1  & 1        \\
		             & 0 & 0 &     & 0 & 1  & 0  & 0  & 1        \\
	      \end{tabular}

	      Each row represents a distinct position and orientation of distinct pentominos

	      \begin{itemize}
		      \item The first 12 (depending on how many pentominos we use) columns should each represent a pentomino shape (eg, X, W, L, ...).
		      \item The remaining columns represent the board in a 1 dimensional way

		            If the board had a state like this:

		            \begin{lstlisting}
						0 1 0
						1 1 1
						1 0 1
					\end{lstlisting}

		            the remaining columns should look like this:

		            \begin{lstlisting}
						0 1 0 1 1 1 1 0 1
					\end{lstlisting}
	      \end{itemize}

	      Each row should contain a single 1 in the firs 12 columns representing which pentomino it is, and 5 (each pentomino fills 5 cells) 1s in the remaining columns.

	\item Actual implementation

	      \begin{enumerate}
		      \item Create matrix A using the Matrix Creator
		      \item Add row numbers to the 0th column of matrix A, ri order to keep track of the rows that we remove
		      \item Use Algorithm X to solve matrix A
		      \item The rows Algorithm X returns contain the position and orientation of the pentominos we have to use
	      \end{enumerate}
\end{enumerate}

\end{document}